\section{Glossary}
\smallskip \hrule height 2pt \smallskip
 
 \begin{itemize}
 	\item \textbf{complementary CDF} (CCDF, or survival function) - can ask how often the random variable is above a particular level.
		Has applications in statistical hypothesis testing, for example, because the one-sided p-value 
			is the probability of observing a test statistic at least as extreme as the one observed.
 	\item \textbf{confidence interval} - An interval that represents the expected range of an estimator if 
		an experiment is repeated many times.
	\item \textbf{effect size} - any statistic that quantitatively measures the strength of a phenomenon.
	\item \textbf{jitter} - random noise added to data for purposes of visualization.  
 	\item \textbf{oversampling} - The technique of increasing the representation of a sub-population in order to avoid errors due to small sample sizes.  
	\item \textbf{p-value} - the \underline{\textbf{p}}robability that, using a given statistical model, the statistical summary 
		(such as the sample mean difference between two compared groups) would be the same as or more extreme 
		than the actual observed results
	\item \textbf{percentile rank} - The percentage of values in a distribution that are less than or equal to a given value.
	\item \textbf{quantizing} - The opposite of smoothing is discretizing, or quantizing.
	\item \textbf{rank} The index where an element appears in a sorted list.
	\item \textbf{raw moment} - A statistic based on the sum of data raised to a power.
		E.g. the correlation between two variables, the regression coefficient in a regression, the mean difference, ...
	\item \textbf{standard error}: - The RMSE of an estimate, which quantifies variability due to sampling error 
		(but not other sources of error).
	\item \textbf{standard score} A value that has been standardized so that it is expressed in standard deviations from the mean.

 \end{itemize}
